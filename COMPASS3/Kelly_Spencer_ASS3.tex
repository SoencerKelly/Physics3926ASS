%! Author = SpencerKelly
%! Date = 2023-03-24

% Preamble
\documentclass[11pt]{article}

% Packages
\usepackage[english]{babel}
\usepackage[utf8x]{inputenc}
\usepackage{amsmath}
\usepackage{graphicx}
\usepackage[colorinlistoftodos]{todonotes}
\usepackage{tcolorbox}
\usepackage{tabto}


\title{\large{PHYS3926 Assignment 3}}
\author{Spencer Kelly}
\date{\today}

\begin{document}
\maketitle
\textbf{Introduction}\\
    \tab Like anything else, stars cannot exist forever, and they too must perish. Though their impermanence is a certainty, how they go out is not always the same. The way in which a star says its goodbyes can be an extravagant, or… less extravagant. If their mass permits it, stars will go out with a bang, literally. Stars with initial masses greater than 8 solar masses will collapse in on themselves and generate a supernova explosion. The stars who are initially not so massive will be able to ‘keep it together’ when facing their demise. These stars are kept from collapsing by something known as electron degeneracy pressure. At their end, these stars can be no more massive than what is known as the Chandrasekhar limiting mass, or else they would supernova. The program to be analyzed in this report serves to calculate the radius and mass of a dying stellar body using a pair of Ordinary Differential Equations (ODEs) which describe the state of the star. The theoretical results are then compared to data from Tremblay et al. (2017).
    \\\\
	\tab With a couple manipulations and choosing accurate assumptions, two ODEs that govern the state of white dwarf stars can be written as equation (1), and equation (2
\\
\textbf{Section Heading}\\
\end{document}